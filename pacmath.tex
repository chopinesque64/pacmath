\documentclass{article}
\usepackage{pacmath}

\title{pacmath}
\begin{document}
A reduction students are likely to make:
\[\pacman \frac{\sin x}{s} = x\,\mathrm{in}\].
Compare to a built-in big operator:
\[\bigoplus \frac{\sin x}{s} = x\,\mathrm{in}\]
The same reduction as an in-line formula:
\(\pacman \frac{\sin x}{s} = x\,\mathrm{in}\).

Test for ``operator-like'' behavior: $\pacman x$ versus 
$\pacman(x)$---does anybody note the difference?
Let us also check that our $\pacman$~symbol does not make the lines further 
apart than usual.  Here it is again:\nobreak\space $\pacman b$.
A few more words to have enough plain lines in the paragraph to make it possible
to compare the leading.  Was that enough?  No, it wasn't: we'd like to get at
least one line further.

Now with limits:
\[
    \pacman_{i=1}^{n} \frac
        {\text{$i$-th magic term}}
        {\text{$2^{i}$-th wizardry}}
\]
And repeated in-line: \( \pacman_{i=1}^{n} x_{i}y_{i} \).

Test for other math styles: subscript~$F_{\!\pacman\alpha}$,
in-line fraction \( \frac{\pacman m}{\pacman n} \),
double superscript \( 2^{2^{\pacman \aleph_{0}}} \)
(this one looks really awkward!).

\begingroup
    \Huge
    Look at the details of the display-style version:
    \[
        \pacman
            \genfrac{<}{>}{0pt}{}
                {\text{something terribly}}{\text{complicated}}
        = 0
    \]
\endgroup

Now we've also got the \texttt{bold} math version:\mathversion{bold}
\[
    \pacman
        \genfrac{<}{>}{0pt}{}
            {\textbf{something terribly}}{\textbf{complicated}}
    = 0
\]
Compare it with \texttt{normal} math\mathversion{normal}:
\[
    \pacman
        \genfrac{<}{>}{0pt}{}
            {\text{something terribly}}{\text{complicated}}
    = 0
\]
In-line math comparison:
{\boldmath $\pacman f(x)$} versus $\pacman f(x)$.

\end{document}
\end{document}

