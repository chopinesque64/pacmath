\documentclass{article}
\usepackage{pacmath}
\usepackage{graphicx}
\usepackage{epigraph}
\usepackage{titling}

\setlength\epigraphwidth{8cm}
\setlength\epigraphrule{0pt}

\title{Pacmath}
\preauthor{} \author{} \postauthor{}
\predate{} \date{} \postdate{}

\newcommand{\bigA}{\DOTSB\bigop[0.92]{\mathrm{A}}}
\newcommand{\bigDelta}{\DOTSB\bigop[1.05]{\Delta}}

\begin{document}
\maketitle
\epigraph{``Let's call this thing Pac-Man."}{--- \textup{Dr. Cook}}
\section{Introduction}
Pacmath finally brings everyone's favorite ad-hoc mathematical symbol to \LaTeX. Don't know what to call an ugly expression you would hate to rewrite again and again? Don't fret, just call it $\pacman$ !

\section{Operator}
$\pacman$ is not a real character; he isn't found in any \TeX{} font. He is a picture with special configuration so he blends in with the crowd. His shape-shifting powers are quite amazing. He has real \TeX{} \textit{bona fides}. He can use his powers to convert arbitrary symbols and pictures to operators. Here, we'll make some new operators and put our friend in there with them:
\[
\sum_{i=1}^n\bigA_{i=1}^n\bigDelta_{i=1}^n\bigpacman_{i=1}^n \cdots.
\]
Slick, eh? We put him in there using the \texttt{\textbackslash bigpacman} command, just like you would use \texttt{\textbackslash bigoplus}.

If that didn't scare you, maybe this will:
\[
\sum_{i=1}^n\bigA_{i=1}^n\bigDelta_{i=1}^n\bigghost_{i=1}^n \cdots.
\]
You can summon a big ghost with \texttt{\textbackslash bigghost}.

Even in inline math mode, the limits still behave as you would expect: $\pacman_{i=1}^n$.

\section{Symbol}
Our friend doesn't always need these superpowers, however. He's also fine going incognito inside a paragraph. Let's test for ``operator-like'' behavior: $f(x)$ versus $\pacman(x)$. He looks like he's just about ready to gobble up that $x$! But here comes Blinky to stop him! $$\ghost(x) \cdots \cdots \pacman(x).$$

\section{Eyes}
Just like you remove numbering from an equation with an asterisk, so you can remove the eyes from our friend and his ghost.
$$\ghost*(x) \cdots \cdots \pacman*(x).$$

\section{Bold}
Now we've also got the \texttt{bold} math version:\mathversion{bold}
\[
    \bigpacman
        \genfrac{<}{>}{0pt}{}
            {\textbf{something terribly}}{\textbf{complicated}}
    = 0
\]
Compare it with \texttt{normal} math\mathversion{normal}:
\[
    \bigpacman
        \genfrac{<}{>}{0pt}{}
            {\text{something terribly}}{\text{complicated}}
    = 0
\]

\section{Edge Cases}
$\pacman$ is at home in all math styles. Just check this out: subscript~$F_{\pacman\alpha}$; in-line fraction \( \frac{\pacman m}{\pacman n} \); double superscript \( 2^{2^{\, \pacman \aleph_{0}}} \). Note that as our friend shrinks, his eye cleanly disappears.

\end{document}
